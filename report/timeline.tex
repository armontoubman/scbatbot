% !TEX root = report.tex

\chapter{Project Timeline}
\label{chap:timeline}

\paragraph{January 2010}
Found an announcement for the AIIDE 2010 StarCraft Competition. Team \massexpand is born.

\paragraph{February 2010 - June 2010}
The Game Programming course is given at the University of Amsterdam and followed by both team members. A proposal to submit the bot as the final project for the course is accepted.

\paragraph{July 2010 - August 2010}
Development and design of the bot starts. For several weeks we have watched countless videos and websites, collecting knowledge about the game. Based on the knowledge, the bot's behaviour is designed to be \emph{emergent}. During initial playtests we noticed its preference to expand, hence the name \massexpand.

\paragraph{September 2010}
The first version of \massexpand is ready. The bot plays well but suffers from slowdowns and crashes.

\paragraph{September 2010 - October 2010}
The competition is held at UC Santa Cruz in California, USA. \massexpand participates in tournament 4 (complete games).

\paragraph{November 2010}
The results are posted online. \massexpand won the first round 1-3, but then lost 0-3 to the bot which would continue to win the tournament. Many of the participants were disqualified due to bugs, which are (rumorly) caused by faulty code provided by the competition organizers.

%\paragraph{February 2011}
%The AIIDE 2011 StarCraft Competition has been announced.

\paragraph{August 2011 - October 2011}
Work continues on a second version. The code is rewritten from scratch and uses the latest version of BWAPI, hoping to eliminate the slowdowns and crashes. This version will be submitted for the Game Programming course.  %Unfortunately, the submission deadline for the next AIIDE 2011 StarCraft competition already passed.

\paragraph{May 2012 - June 2012}
Finishing work on the bot and the documentation. Submission of the report and code.


%%%\paragraph{August 2011}
%%%Work continues on a second version. The code is rewritten from scratch and uses the latest version of BWAPI, hoping to eliminate the slowdowns and crashes. This version will be submitted for the Game Programming course.
%%%
%%%%\paragraph{September 2011}Finishing work on the bot and the documentation.