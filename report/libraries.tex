% !TEX root = report.tex

\chapter{Brood War AI development}
\label{chap:libraries}

StarCraft's internal AI is script-based and not very sophisticated. Since its release, the modding community has been looking for ways to create and use their own AI. One example of such an attempt is the Brood War AI Project\footnote{See \url{http://www.entropyzero.org/BroodwarAI.html}}, which enabled people to create their own scripts and inject them into StarCraft. The scripts allowed macromanagement (building units, general attack orders and managing resources) but not micromanagement (giving orders to individual units). The BWAPI family of libraries solved this problem by giving developers complete access to StarCraft.

\section{BWAPI}
\label{sec:bwapi}

Early 2009, a team of StarCraft enthusiasts released BWAPI (Brood War Application Programming Interface). Unlike the tools that used AI scripts, BWAPI provided programmers with complete access to the game's internals. This was achieved by reverse engineering StarCraft. While this constitutes as a third party hack to the game, and therefore violates the End User License Agreement that comes with StarCraft, Blizzard seems to condone its usage.

BWAPI is a library of classes representing game objects and containing methods for interacting with these game objects. An example class is Unit, which represents units and buildings. This class contains methods for getting information (like the number of hitpoints) and giving orders, such as attack orders for units capable of attacking, and build or research orders for buildings. Of course, not all buildings and units are the same. Information about what units and buildings can and cannot do, and basic information that is the same for units of all types is kept in a class called UnitType. Other examples of included classes are a class representing the players active in a game, and a class that represents the game in general, with methods that for example allow bots to concede and chat with the opponent. More abstract is the Position class, used as a unified notation for positions on the game map, which can be used to coordinate the movement of units and the placement of buildings. The classes from BWAPI are essential building blocks for creating a custom AI.

\section{BWSAL}
\label{sec:bwsal}

The classes and methods in BWAPI are very low level: they are small building blocks with which bots can be constructed. To help developers get started with the creation of new bots, the team behind BWAPI created BWSAL (Brood War Standard Abstraction Layer). BWSAL is an extension of BWAPI, providing classes with more high level functions.

BWSAL consists of classes called managers. Each manager is responsible for a part of a bot's operation. For example, BWSAL has a Build Order Manager, which as the name implies, manages build orders. The managers can bid for control of units, which is assigned by a class called the Arbitrator.

As we will later show, the idea of using managers like this appealed to us but we found BWSAL too constricting in its methodology to use for our own bot. Along with the managers, BWSAL provides two general classes called UnitGroup and BuildingPlacer, which turned out very useful and were incorporated in \massexpand.

\section{BWTA}
\label{sec:bwta}

Besides BWAPI and BWSAL, the team also created BWTA (Brood War Terrain Analyzer). BWTA is an add-on for BWAPI which analyzes StarCraft maps. It can identify base locations, chokepoints (narrow passages which can be of strategic interest) and divide maps into regions. BWTA also provides general functions such as functions to get the distance between two points on the map and functions to determine whether it is even possible to walk between to points on a map (a useful function for maps with islands). BWTA is included with BWAPI by default.

\section{Proxies and Wrappers}

BWAPI, BWSAL and BWTA are written in C++. During the development of these three projects, several side projects were started by third parties to allow the libraries to be used with different programming languages. Two kinds of solutions were made: proxies, that rely on a socket connection to communicate with the libraries, and wrappers for exposing the libraries in different languages. At the time of writing, there are proxies for Java, PHP, Haskell and Ruby, and wrappers for Python, Java, Lua, .NET and Prolog.

During the development of our bot, BWAPI, BWSAL and BWTA were still in development themselves, and so were the proxies and wrappers. New functionality in the libraries had to be ported to the proxies and wrappers before it was accessible in the other languages. We chose to develop our bot in C++ so we were only dependent on the development of BWAPI, and had access to new features as soon as new versions were released. Developing in any of the languages provided by the proxies and wrappers would have added dependency on an extra party to be able to use the latest features. We were not sure how long and how well the proxies and wrappers would be maintained. Therefore, we saw directly using BWAPI with C++ as the best and safest investment of our time.