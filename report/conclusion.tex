% !TEX root = report.tex

\chapter{Conclusion}
\label{chap:conclusion}

In this section, we conclude our project. We mention our thoughts on what went wrong, and how we would have done things differently if we knew what we do now.

At the start of the project, we were quite ambitious. We realized that a computer could have a lot of potential playing StarCraft. Whereas professional players make 400 actions per minute (with an amateur about 100), and the presumption that "there is always something to do" in a match of StarCraft, we hypothesized that a computer could execute a virtually unlimited number of actions per minute. Indeed, at times our bot exceeds 15000 actions per minute. But, its decision making is far from perfect. 

At the time of designing the bot, we had the impression that using AI techniques were absolutely a no-go, because of the complexity and computational challenges. Therefore, we approached the problem by eliciting knowledge. This elicitation in itself was not an easy thing to do. The discussions and explanations on strategies are often of a highly abstract level with little to no quantitative information. Quite the effort was needed to find similarities between strategies and what kind of behaviour is desired for the bot. It was during this process, that we realized that our aspiration to make the perfect StarCraft bot were rather naive. Especially, when we both were still quite inexperienced in C++ at the time, making many implementation errors.

Despite our simplifications, the design and coding of the bot still showed to be very complicated. We had to reorganize our code several times and strategies were altered during the process. Prior to the AIIDE 2010 deadline, we worked on the project excessively, trying to make it in time. The results were announced two months after the deadline. Our bot managed to win the first round, but lost in the second to the team that ended up winning the tournament. We tried to analyze our bot more in depth, but found that the first opponent crashed most of the games, making us win due to disqualification of the opponent. In the end, we only managed to see our bot in action against several human players and the standard built-in AI of StarCraft. It managed to beat the latter several times, but often lost to the human players.

Although the bot was completely revised and completed in the end of 2011, we delayed the completion of the documentation to May 2012. Because we had other courses to attend to, but mostly because both of us were and still are writing our master theses. Needless to say, if we had to choose at this moment, we would definitely had started earlier on finishing this project. Still, we learned a lot during this project: better programming in C++, organizing a lot of code, designing a bot on several levels, and getting more experience with debugging. Furthermore, we have grownFurthermore, we have come to realize that complex projects like this one can not be perfectly solved within a relatively short amount of time.

Looking back on our project with our current knowledge and wisdom, we would have started out programming much differently. Most of the methods we coded were adhoc, and in some cases even duplications. We would also incorporate more elaborated AI techniques, such as data mining replay videos of professional gamers to extract strategies and knowledge. At the time of developing our bot, no such algorithms seemed appropriate for StarCraft. Only much later were several papers published related to direct knowledge extraction in StarCraft. But even with these techniques, it is still requires a lot of effort to develop a bot. To the extent, that we believe it is even harder to design a bot that plays a complicated game, than to design a working new game.

We learned a lot during this project: programming in C++, organizing a lot of code, designing a bot on several levels and a lot of bug testing. Furthermore, we have come to realize that complex projects like this one can not be perfectly solved within a relatively short amount of time.


%reflection, conclusion, results

%Many teams signed up, but at the submission deadline only about 30 teams were ready to participate. 


%The report should be a description of your work during the last lab sessions, describing the game you developed. You should describe what you did, why you did it, what tools you have used, and other background information on the development process. The report is also the place to mention what went 'wrong' and what you would have done differently knowing everything you now at the end of the project. The reflection (introspection) of your project in the report is almost as important as the finished game; It's about the journey, not the destination.


%We started out our project after we came across an Artificial Intelligence (AI) competition for the game; AI2010, a competition meant for bots playing against each other. In this competition, there were four tournaments one could participate in, each tournament having different rules or focus. The first two were focussed on how to control small groups of units, i.e. tactics on how units should work together locally.
%
%The third was a simplified game of StarCraft, were only units and structures of the lowest tier of technology were available. Here, the focus shifts more to placing units appropriately over the map. 
%%% We participate in the AI2010 competition